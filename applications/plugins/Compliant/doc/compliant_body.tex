\newcommand{\vect}[1]{\mathbf{#1}}
\newcommand{\mat}[1]{\mathbf{#1}}

\newcommand{\pos}{\vect{x}}
\newcommand{\xcur}{\vect{x}_{n}}
\newcommand{\xnext}{\vect{x}_{n+1}}
\newcommand{\vel}{\vect{v}}
\newcommand{\vcur}{\vect{v}_{n}}
\newcommand{\vnext}{\vect v_{n+1}}
\newcommand{\force}{\vect{f}}
\newcommand{\forcext}{\vect{f}_{ext}}
\newcommand{\lam}{\vect{\lambda}}
\newcommand{\lcur}{\lam_{n}}
\newcommand{\lnext}{\lam_{n+1}}
\newcommand{\avlam}{\bar{\lam}}
\newcommand{\fcur}{\vect{f}_{n}}
\newcommand{\fnext}{\vect f_{n+1}}
\newcommand{\M}{\mat M}
\newcommand{\Minv}{\mat M^{-1}}
\renewcommand{\P}{\mat P}
\newcommand{\J}{\mat J}
\newcommand{\Jt}{\mat J^T}
\newcommand{\C}{\mat C}
\newcommand{\violation}{ \phi}
\newcommand{\dviolation}{\dot \violation}
\newcommand{\violcur}{\violation_{n}}
\newcommand{\violnext}{\violation_{n+1}}
\newcommand{\dviolnext}{\dot \violation_{n+1}}


\section{Time integration}
Implicit scheme :
\begin{eqnarray}
 \vnext &=& \vcur + h \Minv \left(\alpha \fnext + (1-\alpha) \fcur\right) \\
  \xnext &=& \xcur + h \left( \beta \vnext + (1-\beta) \vcur \right)
\end{eqnarray}
where index $n$ denotes current values while index $n+1$ denotes next values.

\begin{eqnarray}
 \Delta \vel = \vnext -    \vcur  &=& h \Minv \left(\alpha \fnext + (1-\alpha) \fcur\right) \\
\Delta \pos = \xnext -  \xcur    &=& h (v + \beta  \Delta v)
\end{eqnarray}

Constraint violation $\violation$ and its Jacobian $\J$:
\begin{eqnarray}
 \J &=& \frac{\partial \vect \violation}{\partial \pos} \\
 \violnext &\simeq& \violcur + \J \Delta x = \violcur + h   \dviolation + \J h \beta \Delta \vel  \label{eq violnext}\\
\dviolnext &\simeq& \violnext + \J \Delta v \label{eq dviolnext}
\end{eqnarray}


Forces:
\begin{eqnarray}
 \force &=& \forcext + \Jt \lam \\
 \lam_i &=& -\frac{1}{c_i} (  \violation_i + d \dviolation_i ) \label{eq lambda}
\end{eqnarray}
where subscript $i$ denotes a scalar constraint.

Average Lagrange multipliers using equations \ref{eq lambda}, \ref{eq violnext} and \ref{eq dviolnext}:
\begin{eqnarray*}
 \avlam_i &=& \alpha \lnext + (1-\alpha) \lcur \\
&=& -\frac{1}{c_i} ( \alpha \violation + \alpha h \dviolation  + \alpha h \beta \J \Delta \vel + \alpha d \dviolation + \alpha d \J \Delta v + (1-\alpha)\violation + (1-\alpha) d \dviolation  ) \\
&=& -\frac{1}{c_i} ( \violation + d\dviolation + \alpha h \dviolation + \alpha(h\beta+d)\J \Delta v )
\end{eqnarray*}
We can rewrite the previous equation as:
\begin{equation}
 \J \Delta v + \frac{1}{\alpha(h\beta+d)}\C\avlam = - \frac{1}{\alpha(h\beta+d)} (\violation + (d+\alpha h)\dviolation)
\end{equation}
where values without indices denote current values.
The complete equation system is:
\begin{equation}
 \left( \begin{array}{cc}
\frac{1}{h}\P\M & -\P\Jt \\
\ J & \frac{1}{l} \C \end{array}\right)
\left( \begin{array}{c}
\Delta v \\ \avlam
\end{array}\right) = \left( \begin{array}{c}
\P\forcext  \\
- \frac{1}{l} (\violation +(d+\alpha h) \dviolation)
\end{array}\right) 
\end{equation}
where $ l=\alpha(h \beta + d) $
The system is singular due to matrix $\P $, however we can use $ \P \Minv\P $ as inverse mass matrix to compute a Schur complement:
\[ \begin{array}{ccc}
\left( hJPM^{-1}PJ^T + \frac{1}{l}C \right) \bar\lambda &=& -\frac{1}{l} \left(\phi + (d+h\alpha)\dot\phi \right) - hJ M^{-1} f_e \\
\Delta v &=& hP M^{-1}( f_e + J^T \bar\lambda ) \\
\Delta x &=& h( v + \beta \Delta v )
\end{array} \]
