\documentclass[a4paper,11pt]{article}
% LaTeX2Html: setenv TEXINPUTS `pwd`/fig:; latex2html

% ---- graphiques
\usepackage{graphicx}
%\graphicspath{./fig} 
%\usepackage{subfigure}  % several figures in a figure
%\usepackage{pstricks}

% ---- couleurs
% \usepackage[dvips]{color}
% \pagecolor[gray]{1}         % couleur de fond (blanc)

% ---- inclusion de codes

\usepackage{listings}
\lstset{showstringspaces=false,frame=trBL,frameround=tttt,tabsize=4,basicstyle=\tiny,breaklines=true,breakatwhitespace=true}
\lstdefinestyle{bash}{language=bash}
\lstdefinestyle{Perl}{language=Perl}
\lstdefinestyle{C++}{language=C++}
\lstdefinestyle{DTD}{language=XML}
\lstdefinestyle{XML}{language=XML,usekeywordsintag=false,markfirstintag=true}
\newcommand{\includecode}[2]{\lstinputlisting[style=#1]{#2}}
%\lstnewenvironment{code}{}{}
\lstnewenvironment{code_bash}{\lstset{style=bash}}{}
\lstnewenvironment{code_perl}{\lstset{style=Perl}}{}
\lstnewenvironment{code_cpp}{\lstset{style=C++}}{}
\lstnewenvironment{code_dtd}{\lstset{style=DTD}}{}
\lstnewenvironment{code_xml}{\lstset{style=XML}}{}

\newcommand{\textcode}[1]{{\small {\tt #1}}}

% ---- liens entre le dvi g�n�r� et les fichiers sources latex
%\usepackage[active]{srcltx}

% ---- format de page A4
	\setlength{\textwidth }{16cm}	% largeur de ligne
	\setlength{\textheight}{23cm}   % hauteur du texte
	\setlength{\oddsidemargin}{0cm} % marge pages impaires
	\setlength{\evensidemargin}{0cm}% marge pages paires
	\setlength{\topmargin}{0cm} 	
	\setlength{\headheight}{14pt} 
	\setlength{\headsep}{0.5cm} 

%\renewcommand{\floatpagefraction}{1}

% Title Page
\title{SOFA Documentation}
\author{The \sofa~team}
% \author{Fran\c{c}ois Faure\\ J\'er\'emie Allard\\ {\small INRIA Rh\^one-Alpes, Grenoble, France}}

% ------ my custom commands
\newcommand{\sofa}{SOFA }
\newcommand{\todo}[1]{}
\newcommand{\eg}{\textit{e.g.} }
% ---- graphiques
\usepackage[pdftex]{graphicx}
\usepackage{wrapfig}
\usepackage{color}
\usepackage{pst-tree}
%\usepackage{hyperref}

% for accents
\usepackage[latin1]{inputenc}
\usepackage[T1]{fontenc}

\usepackage{algorithm}
\usepackage{algorithmic}

\definecolor{darkgreen}{rgb}{0,0.4,0}
\definecolor{darkblue}{rgb}{0,0,0.4}
\definecolor{darkgray}{rgb}{0.2,0.2,0.2}

% ---- inclusion de codes
\usepackage{listings}
\lstset{showstringspaces=false,tabsize=4,basicstyle=\scriptsize\sffamily,breaklines=true,breakatwhitespace=true,framexleftmargin=5mm, frame=shadowbox, framesep=1pt,rulesepcolor=\color{darkgray},rulesep=.5pt,keywordstyle=\bf\color{blue},commentstyle=\color{magenta},stringstyle=\color{red},numbers=left,numberstyle=\tiny,numbersep=5pt,columns=flexible}

\lstdefinestyle{bash}{language=bash}
\lstdefinestyle{Perl}{language=Perl}
\lstdefinestyle{Python}{language=Python}
\lstdefinestyle{C++}{language=C++,emph={__global__,__shared__,__syncthreads,blockIdx,threadIdx,float3,float4},emphstyle=\bf\color{darkgreen}}
\lstdefinestyle{DTD}{language=XML}
\lstdefinestyle{XML}{language=XML,usekeywordsintag=false,markfirstintag=true}
%begin{latexonly}
\newcommand{\includecode}[2]{
\lstinputlisting[style=#1]{#2}
}
%end{latexonly}


%\lstnewenvironment{code}{}{}
\lstnewenvironment{code_bash}{\lstset{style=bash}}{}
\lstnewenvironment{code_perl}{\lstset{style=Perl}}{}
\lstnewenvironment{code_python}{\lstset{style=Python}}{}
\lstnewenvironment{code_cpp}{\lstset{style=C++}}{}
\lstnewenvironment{code_dtd}{\lstset{style=DTD}}{}
\lstnewenvironment{code_xml}{\lstset{style=XML}}{}

\newcommand{\textcode}[1]{{\sf #1}}




\newcommand{\sofa}{SOFA}
\newcommand{\todo}[1]{}
\newcommand{\eg}{\textit{e.g.} }

\renewcommand{\vec}[1]{\ensuremath{\mathbf{#1 }}} % vector
\newcommand{\Vx}{\vec{x} } % position vector
\newcommand{\Vv}{\vec{v} } % velocity vector
\newcommand{\Va}{\vec{a} } % acceleration vector
\newcommand{\Vf}{\vec{f}} % force
\newcommand{\Vdv}{\vec{\delta\Vv}} % change of velocity vector (unknown in implicit CG, and used in constraint solver
\renewcommand{\P}{\mat{P} } % projection to a constrained space.

\newcommand{\JNL}{\mathbf{\mathcal{J}} }     % mapping des positions
\newcommand{\J}{\mat J }                 % mapping lineaire
\newcommand{\M}{\mat M }             % matrice de masse
\newcommand{\K}{\mat K }             % matrice de raideur
\newcommand{\B}{\mat B }             % matrice d'amortissement
\newcommand{\G}{\mat G }             % jacobien des contraintes



% ---- inclusion de codes
\definecolor{darkgreen}{rgb}{0,0.4,0}
\definecolor{darkblue}{rgb}{0,0,0.4}
\definecolor{darkgray}{rgb}{0.2,0.2,0.2}


% macros mathematiques
\newcommand{\ma}[1]{\ensuremath{\mathbf {#1}}}
\newcommand{\ve}[1]{\ensuremath{\mathbf {#1}}}

\usepackage{amsmath}
\usepackage{amsfonts}
\usepackage{amssymb}

% character styles
\newcommand{\bm}[1]{\ensuremath{\mathbf{{#1}}}}
\newcommand{\mcal}[1]{\mbox{$\mathcal #1$}} % rondes math
\newcommand{\bmcal}[1]{\mbox{\boldmath $\mathcal #1$}} % rondes grasses math
\newcommand{\ensemble}[1]{\mbox{$\mathbb{#1}$}}
\newcommand{\RRR}{\mbox{$\ensemble{R}^3$}} 


% d�finitions
\newcommand{\definition}[2]{\index{#1}{\bf #1}: #2}
\newcommand{\voc}[1]{\index{#1}#1}
\newcommand{\bvoc}[1]{\index{#1}{\bf #1}}

% misc
\newcommand{\EV}[1]{\stackrel{\rightarrow}{#1}}  % espace vectoriel
\newcommand{\EA}[1]{#1}                          % espace affine

% vectors, matrices
%\newcommand{\point}[1]{\mbox{$#1$}}          % un point
\newcommand{\point}[1]{\ensuremath{#1}}          % un point
\newcommand{\mat}[1]{\bm{#1}}         % matrice
\newcommand{\matnm}[3]{\bm{#1_{#2\times #3}}}  % matrice n lignes , m colonnes
\newcommand{\vect}[1]{\bm{#1}}        % vecteur 
%\newcommand{\vecf}[1]{\stackrel{\rightarrow}{#1}}  % vecteur avec fleche
\newcommand{\vecf}[1]{\mbox{$\overrightarrow{#1}$}}  % vecteur avec fleche
\newcommand{\ident}[1]{\bm{I_{#1}}}   % identit� en dimension n
\newcommand{\inv}[1]{#1^{-1}}         % matrice inverse
\newcommand{\psinv}[1]{#1^{+}}        % matrice pseudo-inverse
\newcommand{\transp}[1]{#1^T}         % transpos�e de 1
\newcommand{\trace}[1]{tr(#1)}        % trace
\newcommand{\deter}[1]{\mbox{$|#1|$}}       % determinant
\newcommand{\oppvec}[1]{\mbox{$\left( \vect {#1} \wedge \right)$}}  % operateur matriciel de produit vectoriel

% bases, reperes
\newcommand{\vecin}[2]{\mbox{${}^{#2}#1$}}    % vecteur 1 dans repere 2
\newcommand{\Base}[1]{\ensuremath{\mathcal B_{#1}}} % Symbole du repere 1
\newcommand{\chbase}[3]{\mbox{${}_{#2}^{#3}\mat{#1}$}}  % operateur 1 fait le passage de la base 3 vers la base 2
%\newcommand{\pchbase}[2]{\chbase{\mat{B}}{#1}{#2}}  % matrice de passage de la base 2 vers la base 1
\newcommand{\pchbase}[2]{\chbase{B}{#1}{#2}}  % matrice de passage de la base 2 vers la base 1
\newcommand{\Rep}[1]{\ensuremath{\mathcal R_{#1}}} % Symbole du repere 1
\newcommand{\rep}[1]{\Rep{#1}}                 % Symbole du repere 1
%\newcommand{\pchrep}[2]{\chbase{\mat{F}}{#1}{#2}}  % matrice de passage du repere 1 vers le repere 2, F comme Frame
\newcommand{\pchrep}[2]{\chbase{\bm{C}}{#1}{#2}}  % matrice de passage du repere 2 vers le repere 1

%% Operateur de passage du repere 1 par rapport a 2
%\newcommand{\ChgRep}[2]{\mbox{\boldmath $R_{#1}^{#2}$}}

% rotations	
%\newcommand{\rot}[2]{\mbox{$\mat{R}_{#1,#2}$}}      % rotation vectorielle
\newcommand{\rot}[2]{\ensuremath{\mat{R}_{#1,#2}}}      % rotation vectorielle
\newcommand{\rota}[3]{\mbox{$\mat{R}_{#1,#2,#3}$}}  % rotation affine

% translation
\newcommand{\trans}[2]{\mbox{$\chbase{\vect{t}}{#1}{#2}$}} % passage de #1 vers #2 par une translation, ou translation du repere #2 par rapport au repere #1

% vitesses et acc�l�rations
\newcommand{\VRep}[2]{\mbox{\boldmath $\dot R_{#1}^{#2}$}} % vitesse du repere 1 par rapport a 2 
%\newcommand{\Point}[2]{\mbox{\boldmath ${#1}^{#2}$}}  % Coordonnees d'un point 1 dans un repere 2
\newcommand{\Point}[2]{\mbox{$\vecin{\bm{#1}}{#2}$}}  % Coordonnees d'un point 1 dans un repere 2
\newcommand{\VPoint}[2]{\mbox{\boldmath ${\dot #1}_{/#2}$}} % Vitesse d'un point par rapport � un repere
\newcommand{\APoint}[2]{\mbox{\boldmath ${\ddot #1}_{/#2}$}} % Acceleration d'un point par rapport � un repere

% cinematique du solide
\newcommand{\derivedans}[2]{\mbox{$\dot{#1}^{(#2)}$}}  % derivee du vecteur 1 dans repere 2
\newcommand{\fixedans}[2]{\mbox{$#1_{\in #2}$}}        % vecteur 1 fixe dans repere 2
\newcommand{\vecom}{\mbox{$\bm{\Omega}$}}  % omega de 1 par rapport a 2
\newcommand{\vecrot}[2]{\mbox{$\vecom_{#1/#2}$}}  % omega de 1 par rapport a 2
\newcommand{\accrot}[2]{\mbox{$\dot{\vecom}_{#1/#2}$}}  % omega de 1 par rapport a 2
\newcommand{\vfdans}[3]{\mbox{$\vec V^{#2/#3}_{#1}$}}    % vitesse de 1 fixe dans 2 par rapport a 3
\newcommand{\afdans}[3]{\mbox{$\vec \Gamma^{#2/#3}_{#1}$}}    % acceleration de 1 fixe dans 2 par rapport a 3
\newcommand{\vmdans}[2]{\mbox{$\vec V^{/{#2}}_{#1}$}}    % vitesse de 1 mobile dans 2
\newcommand{\amdans}[2]{\mbox{$\vec \Gamma^{/#2}_{#1}$}}    % acceleration de 1 mobile dans 2

% chaines articulees
\newcommand{\liaison}[2]{\mbox{$\mathcal L_{#1,#2}$}}  % liaison du pere 1 vers fils 2 (et repere intermediaire)
\newcommand{\liaisonprime}[2]{\mbox{$\mathcal L'_{#1,#2}$}}  % deuxieme repere intermediaire de la liaison du pere 1 vers fils 2
\newcommand{\liaisonP}[2]{\mbox{$\mathcal L_{#1,#2}$}}  % Repere dans pere 1 de la liaison vers fils 2 
\newcommand{\liaisonC}[2]{\mbox{$\mathcal L'_{#1,#2}$}}  % Repere dans fils de la liaison du pere 1 vers fils 2 
%\newcommand{\transP}[2]{\pchrep{\liaisonP{#1}{#2}}{#1}}  % Matrice du repere dans pere de la liaison du pere 1 vers fils 2 
%\newcommand{\transC}[2]{\pchrep{\liaisonC{#1}{#2}}{#2}}  % Matrice du repere dans pere de la liaison du pere 1 vers fils 2 
%\newcommand{\transPC}[2]{\pchrep{\liaisonC{#1}{#2}}{\liaisonP{#1}{#2}}}  % matrice de passage entre repere liaison dans fils et repere de liaison dans pere
\newcommand{\transP}[2]{\chbase{C_p}{#2}{#1}}  % Matrice du repere dans pere de la liaison du pere 1 vers fils 2 
\newcommand{\transC}[2]{\chbase{C_c}{#2}{#1}}  % Matrice du repere dans pere de la liaison du pere 1 vers fils 2 
\newcommand{\transPC}[2]{\chbase{C_l}{#2}{#1}}  % matrice de passage entre repere liaison dans fils et repere de liaison dans pere
% \pchrep{fils}{pere} = \liaisonP{pere}{fils}\deplPC{pere}{fils}\liaisonC{pere}{fils}


\newcommand{\pctab  }{\hspace{0.15in}      }  % Pseudo-code indentation.
\newcommand{\code}[1]{ 
\begin{makeimage}
\begin{tabbing} \pctab \= \pctab \= \pctab \= \pctab \= \pctab \= \pctab \= \pctab \kill
#1
\end{tabbing}
\end{makeimage}
}


\begin{document} 
\maketitle

% \begin{abstract}
% This is the documentation of the SOFA library. Chapter~\ref{chapter:pba} gives theoretical background on physically-based animation. Chapter~\ref{chapter:as} shows how to implement simulations using \sofa. Chapter~\ref{chapter:es} describes how to integrate new components in \sofa.
% \end{abstract}

\tableofcontents


\section{Introduction}
\sofa is an open-source C++ library for physical simulation, primarily targeted to medical simulation.
It can be used as an external library in another program, or using one of the associated GUI applications.

The main feature of \sofa  compared with other libraries is its high flexibility. It allows the use of multiple interacting geometrical models of the same object, typically, a mechanical model with mass and constitutive laws, a collision model with simple geometry, and a visual model with detailed geometry and rendering parameters. Each model can be designed independently of the others. During run-time, consistency is maintained using mappings.


Additionally, \sofa  scenes are modeled using a data structure similar to hierarchical scene graphs commonly used in graphics libraries. This allows the splitting of the physical objects into collections of independent components, each of them describing one feature of the model, such as mass, force functions and constraints.
For example, you can replace spring forces with finite element forces by simply replacing one component with another, all the rest (mass, collision geometry, time integration, etc.) remaining unchanged.

Moreover, simulation algorithms, such as time integration or collision detection and modeling, are also modeled as components in the scene graph. This provides us with the same flexibility for algorithms as for models.

Flexibility allows one to focus on its own domain of competence, while re-using the other's contributions on other topics. 
However, efficiency is a major issue, and we have tried to design a framework which allows both efficiency and flexibility.

\section{Commented example}
Figure~\ref{fig:mixedPendulum} shows a simple scene composed of two different objets, one rigid body and one particle system, and linked by a spring.
\begin{figure}
 \centering
 \includegraphics[width=0.9\linewidth]{mixedPendulum.png}
 \caption{A pendulum composed of a rigid body (reference frame and yellow point) attached to an elastic string (green) fixed at one end (pink point). The corresponding scene graph is displayed on the left.}
 \label{fig:mixedPendulum}
\end{figure}
This scene is modeled and simulated in C++ as shown in appendix~\ref{cpphybrid}.

\todo{Graphviz scene graph}

The scene is modeled as a tree structure with four nodes:
\begin{itemize}
 \item \texttt{root} 
 \item \texttt{deformableBody} corresponds to the elastic string
 \item \texttt{rigidBody} corrsponds to the rigid object
 \item \texttt{rigidParticles} corresponds to a set of particles (only one in this case) attached to the rigid body
\end{itemize}
Each node can have children nodes and \textit{components}. Each component implements a reduced set of functionalities.


One of the most important type of component is the \texttt{MechanicalObject}, which contains a list of \textit{degrees of freedom} (DOF), i.e. coordinates, velocities, and associated auxiliary vectors such as forces and accelerations. 
All the coordinates in a \texttt{MechanicalObject} have the same type, e.g. 3D vectors particles, or (translation, rotation) pairs for rigid bodies. \texttt{MechanicalObject}, like many other \sofa classes, is a generic (C++ template) class instantiated on the types of DOF it stores.
The particle DOFs are drawn as white points, whereas the rigid body DOFs are drawn as red, green, blue reference frame axes.
There can be at most one \texttt{MechanicalObject} attached to a given node. This guarantees that all the components attached to the same node process the same types of DOF. Consequently, the particles and the rigid body necessarily belong to different nodes. 

In this example, the masses are stored in \texttt{UniformMass} components.
The types of their values are related to the types of their associated DOF.
\texttt{UniformMass} is derived from the abstract \texttt{Mass} class, and stores only one value, for the case where all the associated objects have the same mass. If necessary, it can replaced by a \texttt{DiagonalMass} instanciated on the same DOF types, for the case where the associated objects have different masses. This is an important feature of \sofa: each component can be replaced by another one deriving from the same abstract class and instanciated on the same DOF types. This results in a high flexibility.

The \texttt{FixedConstraint} component attaches a particle to a fixed point in world space, drawn in pink. The constraints act as filters which cancel the forces and displacements applied to their associated particle(s). They do not model more complex constraints such as maintaining three points aligned.

The \texttt{StiffSpringForceField} stores a list of springs, each of them modeled by a pair of indices, as well as the standard physical parameters, stiffness, damping and rest length.

The rigid body is connected to the deformable string by a spring.
Since this spring is shared by the two bodies, it is modeled in the \texttt{StiffSpringForceField} attached to a common ancestor, the graph root in this example.
Our springs can only connect particles. We thus need to attach a particle to the rigid body. Since the particle DOFs types are different from the rigid body DOF types, they have to be stored in another \texttt{MechanicalObject}, called \texttt{rigidParticleDOF} in this example, and attached to a different node. 
However, \texttt{rigidParticleDOF} is not a set of independent DOF, since they are fixed in the reference frame of the rigid body. We thus attach it to a child node of the rigid body, and connect it to \texttt{rigidDOF} using a \texttt{RigidMapping}. This component stores the coordinates of the particle in the reference frame of the rigid body. Its task is to propagate the position, velocity and displacement of the rigid body down to the yellow particle, and conversely, to propagate the forces applied to the particle up to the rigid body.

Mappings are one of the major features of \sofa. They allow us to use different geometric models for a given body, e.g. a coarse tetrahedral mesh for viscoelastic internal forces, a set of spheres for collision detection and modeling, and a fine triangular mesh for rendering.

The gravity applied to the scene is modeled in the \texttt{Gravity} component near the root. It applies to all the scene, unless locally overloaded by another gravity component inside a branch of the tree.

So far, we have discussed the physical model of the scene. 
To animate it, we need to solve an \textit{Ordinary Differential Equation} (ODE) in time. 
There are plenty of ODE solvers, and \sofa allows the design and the re-use of a wide variety of them. 
Here we use a simple explicit Euler method, modeled using an \texttt{EulerSolver} component.
It triggers computations such as force acumulation, acceleration computation and linear operations on state vectors.
More sophisticated solvers are available in \sofa, and can be used by simply replacing the  \texttt{EulerSolver} component by another one, e.g \texttt{RungeKutta4} or \texttt{CGImplicit}.

Other capabilities of \sofa, such as collision detection and response, will be discussed in subsequent sections.

\section{Actions}

\section{Mappings}

\section{ODE Solvers}

\section{Collision detection}

\section{Constraint-based methods}

\section{File format}

\section{Limitations}

\pagebreak
\appendix
\section{C++ code of the hybrid pendulum}\label{cpphybrid}
\includecode{C++}{../applications/tutorials/mixedPendulum/Main.cpp}

% \chapter{Animation in \sofa}\label{chapter:as}
This section shows how the material presented in chapter \ref{chapter:pba} is implemented in \sofa. A class diagram of the mechanical core is presented in appendix \ref{sec:umlmeca}.
\section{Particles}
\subsection{Simple example}
Here we simulate two particles subject to gravity. An introduction to particle dynamics is given in section \ref{sec:particles}. 
Figure \ref{fig:singleParticleCollaboration} shows the collaboration diagram and the code used to model a set of particles subject to gravity. The code is available in project {\tt doc/src\_examples/example1}.


\begin{figure}[htp]
\begin{center} \begin{tabular}{cc}
\begin{minipage}[b]{6cm}
	\includegraphics*[width=6cm]{fig/singleParticleCollaboration.eps}  
	%\input{fig/singleParticleCollaboration.tex} 
\end{minipage}
 &
\begin{minipage}[b]{9cm} 
\begin{code_cpp}
#include "Sofa/Components/Scene.h"
#include "Sofa/Components/MassObject.h"
#include "Sofa/Components/EulerSolver.h"
#include "Sofa/GUI/FLTK/Main.h"

using namespace Sofa::Components;
using namespace Sofa::Core;
using namespace Sofa::GUI::FLTK;
typedef Sofa::Components::Common::Vec3Types MyTypes;
typedef MyTypes::Deriv Vec3;
        
int main(int argc, char** argv) 
{
    Scene* scene = new Scene;
    scene->setDt(0.04);
    
    MechanicalGroup* group = new MechanicalGroup;
    scene->addBehaviorModel(group);
    group->setSolver( new EulerSolver );
    
    MassObject<MyTypes>* particles = new MassObject<MyTypes>;
    group->addObject(particles);
    scene->addVisualModel(particles);                 
    particles->setGravity( Vec3( 0,-1,0 ) );
    particles->addMass( Vec3(2,0,0), Vec3(0,0,0), 1 ); 
    particles->addMass( Vec3(3,0,0), Vec3(0,0,0), 1 );


    scene->init();
    MainLoop(argv[0]);
    return 0;
}
\end{code_cpp}
\end{minipage}
\end{tabular}
\end{center}
\label{fig:singleParticleCollaboration} 
\caption{Collaboration diagram and code for the animation of free particles.}
\end{figure}


% 
% \chapter{Extending Sofa}

This chapter presents how different types of classes can be added to \sofa{} to implement new behaviors.

\section{Adding a new DynamicObject}

A DynamicObject is responsible for implementing the behavior of a given type of body. It contains the degrees of freedoms (\textit{DOFs}) of the body (within the MechanicalObject class), as well as its mass (i.e. how the body moves given an applied force).

To add a new DynamicObject in \sofa, the following steps are required:

\begin{enumerate}
\item Specify the type of its DOFs in a \textit{DataTypes} class.
\item Create the \textit{DynamicObject} derived class implementing the behaviors of the body, or instanciate an existing class with you \textit{DataTypes} class if already available.
\item Create a method to create an instance of the body given an XML node.
\item Register the new DynamicObject in the Sofa::Components::XML::DynamicNode::Factory factory.
\item Create a XML scene file containing a DynamicModel with the \textit{type} of our class.
\item Have fun!
\end{enumerate}

If the new class will only be created procedurally, then only the first two steps are required.

We will now details these steps using the example of an object containing a set of one-dimensional particles.

The code corresponding to this example is available in the following files:
\begin{description}
\item[Sofa/doc/src\_examples/example3/MassObject1d.h]~\\
 Declaration of MassObject1d.
\item[Sofa/doc/src\_examples/example3/MassObject1d.cpp]~\\
 Creation from XML and registration in the Factory
\item[Sofa/doc/src\_examples/example3/Main.cpp]~\\
 Main program
\item[Sofa/doc/src\_examples/example3/test1.scn]~\\
 Test scene file
\end{description}

\subsection{Specification of the Degrees of Freedom}\label{sec:DOF}

The data types used by the body are specified in a class containing the following definitions:

\begin{description}
\item[Coord]~\\
 Type of DOFs.
\item[Deriv]~\\
 Type of derivatives (velocity, forces, displacements).
\item[VecCoord]~\\
 Container of DOFs.
\item[VecDeriv]~\\
 Container of derivatives.
\item[void set(Coord\& c, double x, double y, double z)]~\\
 Utility method to set the value of a point.
\item[void add(Coord\& c, double x, double y, double z)]~\\
 Utility method to add a value to a point (for applyTranslation).

\end{description}

In our example, we can use double as the type of DOFs and std::vector as containers:

\begin{verbatim}
class Vec1dTypes
{
public:
  typedef double Coord;
  typedef double Deriv;
  typedef std::vector<Coord> VecCoord;
  typedef std::vector<Deriv> VecDeriv;
  
  /// Here we only use the first coordinate
  static void set(Coord& c, double x, double /*y*/, double /*z*/)
  {
    c = x;
  }
  
  static void add(Coord& c, double x, double /*y*/, double /*z*/)
  {
    c += x;
  }
};
\end{verbatim}

\subsection{Body Behaviors Implementation}

In \sofa{} a DynamicObject does not implement the time integration algorithm, which is handled by an separate Solver. Instead, it implements basic operations, such as compute forces, combine several vectors, etc. Of these operations, most are only dependant on the types of DOFs, or are computed through external classes (ForceFields for instance). Only 3 operations related to the mass remain to be implemented by DynamicObject subclasses:

\begin{description}
\item[computeForce] must be modified to add gravity ( $\ve f = \ve f + \ma M \ve g$ ).
\item[accFromF] must be implemented to convert forces to accelerations ( $\ve a = \ma M^{-1} \ve f$ ).
\item[addMDx] must be implemented to multiply a given deplacement by the mass ( $ \ve r = \ve r + \ma M \ve dx$ ).
\end{description}

In our example, a class \textit{MassObject} already exists to handle object represented as a set of particles. So we just need to instanciate it to the type of DOFs we want to use:

\begin{verbatim}
typedef MassObject<Vec1dTypes> MassObject1d;
\end{verbatim}

As a reference, here is how MassObject implements the 3 operations mentioned earlier:

\begin{verbatim}
template <class DataTypes>
void MassObject<DataTypes>::addMDx(VecDeriv* res, VecDeriv* dx)
{
  for( unsigned i=0; i<dx->size(); ++i )
    (*res)[i] += (*dx)[i] * masses[i].mass;
}


template <class DataTypes>
void MassObject<DataTypes>::accFromF(VecDeriv* a, VecDeriv* f)
{
  a->resize(f->size());
  for( unsigned i=0; i<f->size(); ++i )
    (*a)[i] = (*f)[i] / masses[i].mass;
}


template <class DataTypes>
void MassObject<DataTypes>::computeForce(VecDeriv* result)
{
  Inherit::computeForce(result);
  // Add Gravity
  for (unsigned int i=0;i<result->size();i++)
  {
    (*result)[i]+=gravity*masses[i].mass;
  }
}
\end{verbatim}

\subsection{Instanciation from XML}

\sofa{} implements a mechanism to load a scene from a XML file using a two step process:

\begin{enumerate}
\item The XML file is parsed and converted to a tree of Node class.
\item Each node use a Factory to instanciate the described object.
\end{enumerate}

A DynamicObject is described by a Sofa::Components::XML::DynamicNode. It contains the type of the class to instanciate, as well as a set of string attributes. To build our DynamicObject from this description we need a function to construct and configure a new instance given the pointer to the XML::DynamicNode. This can either be a constructor in our class, or an external function with the following prototype:
\begin{verbatim}
namespace Sofa { namespace Components { namespace Common {
void create(MyDynamicObject*& obj, XML::Node<Sofa::Abstract::DynamicModel>* arg);
} } }
\end{verbatim}
where \textit{MyDynamicObject} is the name of our new class.

\textbf{Note:} the inclusion in the Sofa::Components::Common namespace is necessary for the Factory class to find our function. If anyone find a better design removing this requirement please tell me!

Typically, the object will either be constructed from attributes given in the XML node, or using an external description file.

Our 1D particles example is simple enough not to require an external file. We can implement a creation function as follow:

\begin{verbatim}
/// Read a vector of scalars from a string.
void readVec1(std::vector<double>& vec, const char* str)
{
  vec.clear();
  if (str==NULL) return;
  const char* str2 = NULL;
  for(;;)
  {
    double v = strtod(str,(char**)&str2);
    std::cout << v << std::endl;
    if (str2==str) break;
    str = str2;
    vec.push_back(v);
  }
}

namespace Sofa { namespace Components { namespace Common {
/// Construct a MassObject1d object from a XML node.
void create(MassObject1d*& obj, XML::Node<Sofa::Abstract::DynamicModel>* arg)
{
        obj = new MassObject1d();
  obj->clear();
  std::vector<double> mass;
  std::vector<double> pos;
  std::vector<double> vel;
  std::vector<double> fixed;
  readVec1(mass,arg->getAttribute("mass"));
  readVec1(pos,arg->getAttribute("position"));
  readVec1(vel,arg->getAttribute("velocity"));
  readVec1(fixed,arg->getAttribute("fixed"));
  if (arg->getAttribute("gravity"))
  {
    obj->setGravity(atof(arg->getAttribute("gravity")));
  }
  unsigned int maxsize = mass.size();
  if (pos.size()>maxsize) maxsize = pos.size();
  if (vel.size()>maxsize) maxsize = vel.size();
  double defaultmass = (mass.empty()?1.0:*mass.rbegin());
  while (mass.size()<maxsize)
    mass.push_back(defaultmass);
  double defaultpos = 0;
  if (!pos.empty()) defaultpos = *pos.rbegin();
  while (pos.size()<maxsize)
    pos.push_back(defaultpos);
  double defaultvel = 0;
  if (!vel.empty()) defaultvel = *vel.rbegin();
  while (vel.size()<maxsize)
    vel.push_back(defaultvel);
  for (unsigned int i=0;i<maxsize;i++)
  {
    obj->addMass(pos[i], vel[i], mass[i], 0.0,
      (std::find(fixed.begin(), fixed.end(), (double)i)!=fixed.end()));
  }
} } } }
\end{verbatim}

Note that this function is quite long, as it construct vectors of values (masses, positions, velocities) from strings. For simpler cases, where only a filename is required for instance, generic creation functions are provided:
\begin{verbatim}
namespace Sofa { namespace Components { namespace Common {
/// Construct a MassObject1d object from a XML node using an external file.
void create(MassObject1d*& obj, XML::Node<Sofa::Abstract::DynamicModel>* arg)
{
  XML::createFromFilename(obj, arg);
} } } }
\end{verbatim}
However it is then necessary to implement the external file loading method.

\subsection{Factory Registration}

Once all functionnalities are implemented, it is necessary to register the new class in the Sofa::Components::XML::DynamicNode::Factory factory in order for \sofa{} to know about it. This requires adding the following "magic" line:

\begin{verbatim}
Creator< XML::DynamicNode::Factory, MassObject<Vec1dTypes> >
  MassObject1dClass("MassObject1d");
\end{verbatim}

This command will register our class to the Factory during the initialization of the program. For this to work we must still ensure our code is linked in the final binary. To do this we must add the following line in the .cpp file containing the Creator command:
\begin{verbatim}
SOFA_DECL_CLASS(MassObject1d)
\end{verbatim}
and then add a corresponding line in a .cpp file used during the execution of the program (such as the file containing the main() function, or in Sofa/Components/init.cpp if the new class is integrated in \sofa{}):
\begin{verbatim}
SOFA_INIT_CLASS(MassObject1d)
\end{verbatim}
This last step is required to work around portability issues, and might be removed if a better solution is found.

When the given type name is used in an XML file, the Factory will now be able to construct our custom DynamicObject.

\subsection{Testing}

Our new class can now be loaded by writing a small XML file:

\begin{verbatim}
<Scene dt="0.005" showBehaviorModels="1" showCollisionModels="1" showMappings="1" showForceFields="1">
	<Group>
		<Solver type="RungeKutta4"/>
		<DynamicModel type="MassObject1d" name="M1" position="0 1 2 3 4 5" fixed="5" gravity="-9.8"/>
	</Group>
</Scene>
\end{verbatim}

To run it, go to Sofa/doc/src\_examples/example3 and execute
\begin{verbatim}
./run test1.scn
\end{verbatim}

\section{Adding a new Mapping}

Many existing objects in \sofa{} expect to work with 3D particles. To be able to use them with our 1D particles, we can create a MechanicalMapping which will convert informations between the two representations.

Adding a new Mapping in \sofa{} requires the following steps:

\begin{enumerate}
\item Create the \textit{Mapping} or \textit{MechanicalMapping} derived class computing the mapping from the input model to the ouput, and accumulating forces back for MechanicalMappings.
\item Create a method to create an instance of the mapping given an XML node.
\item Register the new mapping in the Sofa::Components::XML::MappingNode::Factory factory.
\item Create a XML scene file containing a Mapping with the \textit{type} of our class.
\item Have more fun!
\end{enumerate}

We will now detail these steps using the 1D -\> 3D mechanical mapping example.
The code for this mapping is available in the \textbf{Sofa/doc/src\_examples/example3/LinearMapping.cpp} file.

\subsection{Mapping Implementation}

A MechanicalMapping is used in the scene to link two MechanicalObjects, an input model and an output model. Positions, velocities and displacements are propagated from the input model to the output, and forces are accumulated in the other direction. This can be implemented by overloading 3 methods:

\begin{description}
\item[apply] compute output positions from input ones
\item[applyJ] compute output derivatives (velocity of displacement) from input ones
\item[applyJT] accumulate back output forces (or df) into input ones
\end{description}

If the mapping is linear these operations can be expressed in terms of the mapping matrix $\ma J$: apply and applyJ are equivalent to $\ve out = \ma J \ve in$, and applyJT is equivalent to $ \ve in += \ma J^{t} \ve out $.

A scene in \sofa{} can contain mechanical models with different types of degrees of freedoms (see section~\ref{sec:DOF}). The mapping can either be generic relatively to the types used in the input and output models, or requires them to use specific types. The first solution requires the mapping implementation to be declared as a template of the type of DOFs, while the second solution requires implementing a "standard" class.

For our example, we will create a non-templated mapping for simplicity, although a templated version would be very similar.

\begin{verbatim}
class LineMapping : public Sofa::Core::MechanicalMapping< Sofa::Core::MechanicalObject<Vec1dTypes>, Sofa::Core::MechanicalObject<Vec3dTypes> >
{
public:
  // Simplified notation for all involved classes
  typedef Sofa::Core::MechanicalMapping< Sofa::Core::MechanicalObject<Vec1dTypes>, Sofa::Core::MechanicalObject<Vec3dTypes> > BaseMapping;
  typedef BaseMapping::In In;
  typedef BaseMapping::Out Out;
  typedef Out::VecCoord VecCoord;
  typedef Out::VecDeriv VecDeriv;
  typedef Out::Coord Coord;
  typedef Out::Deriv Deriv;

  Coord p0; ///< Origin of the 3D line
  Deriv dx; ///< Direction of the 3D line
  
  LineMapping(In* from, Out* to, const std::string& /*name*/)
  : BaseMapping(from, to), p0(0,0,0), dx(1,0,0)
  {
  }
  
  void apply( Out::VecCoord& out, const In::VecCoord& in )
  {
    out.resize(in.size());
    for(unsigned int i=0;i<out.size();i++)
      out[i] = p0+dx*in[i];
  }
  
  void applyJ( Out::VecDeriv& out, const In::VecDeriv& in )
  {
    out.resize(in.size());
    for(unsigned int i=0;i<out.size();i++)
      out[i] = dx*in[i];
  }
  
  void applyJT( In::VecDeriv& out, const Out::VecDeriv& in )
  {
    for(unsigned int i=0;i<out.size();i++)
      out[i] += dx*in[i];
  }
};
\end{verbatim}

\subsection{Instanciation from XML and Registration in Factory}

This process is identical to the corresponding steps in the DynamicObject case, except that the XML Node to use is XML::MappingNode instead of XML::DynamicNode.

In our example, the additional code required for this step is:
\begin{verbatim}

namespace Sofa { namespace Components { namespace Common {

void create(LineMapping*& obj, XML::Node<Sofa::Core::BasicMapping>* arg)
{
  XML::createWith2Objects< LineMapping, LineMapping::In, LineMapping::Out>(obj, arg);
  if (obj!=NULL)
  {
    obj->p0[0] = atof(arg->getAttribute("x0","0"));
    obj->p0[1] = atof(arg->getAttribute("y0","0"));
    obj->p0[2] = atof(arg->getAttribute("z0","0"));
    obj->dx[0] = atof(arg->getAttribute("dx","1"));
    obj->dx[1] = atof(arg->getAttribute("dy","0"));
    obj->dx[2] = atof(arg->getAttribute("dz","0"));
  }
} } } }

SOFA_DECL_CLASS(LineMapping)

Creator< XML::MappingNode::Factory, LineMapping > LineMappingClass("LineMapping", true);
\end{verbatim}

Note the true argument in the Creator command. It means that other classes with the same type name are authorized in the Factory, implementing the same mapping for other datatypes for instance.

\subsection{Testing}

Our new class can now be used to add a 3D spring force field to our 1D masses by writing a small XML file:

\begin{verbatim}
<Scene dt="0.005" showBehaviorModels="1" showCollisionModels="1" showMappings="1" showForceFields="1">
	<Group>
		<Solver type="RungeKutta4"/>
		<DynamicModel type="MassObject1d" name="M1" position="0 1 2 3 4 5" fixed="5" gravity="-9.8">
		<MechanicalModel type="Vec3d" name="Points">
		<ForceField type="StiffSpringForceField" filename="test2.xs3"/>
		</MechanicalModel>
		<Mapping type="LineMapping" object1=".." object2="Points" />
		</DynamicModel>
	</Group>
</Scene>
\end{verbatim}

To run it, go to Sofa/doc/src\_examples/example3 and execute
\begin{verbatim}
./run test2.scn
\end{verbatim}

The same mapping can also be used to attach a collision model:

\begin{verbatim}
<Scene dt="0.005" showBehaviorModels="1" showCollisionModels="1" showMappings="1" showForceFields="1">
	<CollisionPipeline>
		<CollisionDetection name="N2" type="BruteForce" />
		<Contact name="Response" contact="default" />
		<CollisionGroup name="Group" />
	</CollisionPipeline>
	<Group>
		<Solver type="RungeKutta4"/>
		<DynamicModel type="MassObject1d" name="M1" position="0 1 2 3 4 5" fixed="5" gravity="-9.8">
		<MechanicalModel type="Vec3d" name="Points">
		<ForceField type="StiffSpringForceField" filename="test2.xs3"/>
		</MechanicalModel>
		<Mapping type="LineMapping" object1=".." object2="Points" />
		<CollisionModel type="Sphere" name="Spheres" filename="test3.sph"/>
		<Mapping type="LineMapping" object1=".." object2="Spheres" />
		</DynamicModel>
	</Group>
</Scene>
\end{verbatim}

To run it, go to Sofa/doc/src\_examples/example3 and execute
\begin{verbatim}
./run test3.scn
\end{verbatim}

You can now pick the particles with the mouse by pressing the shift key and left mouse button.

% 
% \chapter{Physically-based animation} \label{chapter:pba}
\section{Particles} \label{sec:particles}
\subsection{Basic equation}
A particle is a moving point with a mass. Generally the mass is a constant over time while its position and velocity can vary. We therefore consider the mass $ m $ (in $ kg $) as an attribute while its position $ x $ (in $ m $) and its velocity $ v $ (in $m.s^{-1}$) are its state variables. 
Particles obey Newton's law: 
\begin{equation} \label{eq:newton}
ma = \Sigma f
\end{equation} 
where $a$ is the particle's acceleration (in $m.s^{-2}$). This defines the Ordinary Differential Equation (ODE) $a(t) = f(x,v,t)$. Physically-based animation requires us to repeatedly solve it over the interval $[t,t+dt]$ and redisplay.

From equation \ref{eq:newton} we can deduce $a=1/m\;\Sigma f$ and integrate time over a time step dt, and so on. The simplest integration method is Euler's explicit scheme:
\begin{eqnarray}
v(t+dt) &=& v(t)+a(t)dt \nonumber \\
x(t+dt) &=& x(t)+v(t)dt \label{eq:expliciteuler}
\end{eqnarray}

Particles can live in any k-dimensional space $\Re^k$, where state variables and forces are k-dimensional vectors of scalar values. The dynamics equation \ref{eq:newton} applies to each scalar value.
When considering $n$ particles the equations can be conveniently written in matrix form:
\begin{eqnarray*}
\ma M \ve a &=& \ve f\\
\ve a &=& \ma M^{-1} \ve f
\end{eqnarray*}
where \ma M is a diagonal matrix of dimension $kn\times kn$ , \ve a and \ve f are vectors of dimension $kn$ gathering all the scalar components associated with each particle.

Numerical integration can generate instabilities leading to the divergence of the simulated system. To avoid this, one solution is to decrease the time step. The other solution is to use an implicit integration scheme, which takes into account the variation of the forces during the time step. The simplest one is the implicit Euler's method which applies a step based on the forces at the end of the time step instead of the beginning. Equation \ref{eq:expliciteuler} becomes:
\begin{eqnarray}
v(t+dt) &=& v(t)+a(t+dt)dt \nonumber\\
x(t+dt) &=& x(t)+v(t+dt)dt \label{eq:impliciteuler}
\end{eqnarray}
Since $a(t+dt)$ is unknown we have to solve an equation.
If we write $v(t+dt) = v(t)+\Delta v$ the matrix form equation to solve is:
\begin{equation}
\label{eq:matimplicit}
\left( \ma M + dt \ma D + dt^2 \ma K \right) \ve{\Delta v} = dt \left( \ve f(t) + \ma D \ve{v}(t) \right) 
\end{equation}
where the damping matrix $\ma D = \delta \ve f/\delta \ve v$ encodes the variation of force given a variation of velocity, and the stiffness matrix $\ma K = \delta \ve f/\delta \ve p$ encodes the variation of force given a variation of position.

\subsection{Forces}
The forces are responsible for the accelerations of the bodies. Their physical unit is the Newton ($N=kg.m.s^{-2}$).
Here we briefly review the most commonly used forces.

\subsubsection{Weight}
A uniform gravitational field $g$ (in $m.s^{-2}$) applies a force 
\begin{equation} \label{eq:gravity}
f = mg
\end{equation} 
to each particle where $m$ is the mass of the particle.

\subsubsection{Linear damping}
Damping transforms kinetic energy to heat by applying a force opposed to the velocity. It tends to slow down the objects. Linear damping is proportional to the velocity, thus 
\begin{equation}\label{eq:lineardamping}
f=-\nu v
\end{equation}
where $v$ is the velocity of the particle and $\nu$ a positive scalar (in $kg.s^{-1}$).

\subsubsection{Air damping}
Air damping is proportional to the square of the velocity of the body with respect to the air:
\begin{equation}
f = -\rho S_u C_u v^2 u 
\end{equation}
where $\rho$ is the volumic mass the air ($kg.m^{-3}$), $v$ the velocity of the object, $u$ a no-dimensional unit vector in the direction of the velocity, $S$ the area ($m^2$) of the object projected along $u$, and $C_u$ a no-dimensional coefficient associated with the shape of the object and the direction $u$.

\subsubsection{Linear springs}
A springs applies an elastic force between two points. It is modeled using its rest length $l_0$ (in $m$) and its stiffness $k$ (in $N.m^{-1}$). 
Let $i$ and $j$ be the indices of points linked by a given spring. 
The force applied by a linear spring to point $i$ is given by:
\begin{equation}
\label{eq:spring}
f_i = k( l-l_0 ) u
\end{equation}
where $l=\|x_j - x_i\|$ is the distance between the points and $u$ a unit vector pointing from point $i$ to point $j$. The force $f_j$ applied to point $j$ is the opposite: $f_j = -f_i$. Nonlinear springs can be used to model more complex behaviors.

\subsubsection{Linear damped springs}
Damping forces are commonly associated with springs in order to dissipate energy. They are opposed to the relative velocity of the points. They are typically modeled using a coefficient $\nu$ (in $kg.s^{-1}$).
The force applied by a linear damped spring to point $i$ is given by:
\begin{equation}
\label{eq:dampedspring}
f_i = \left( k( l-l_0 ) + \nu v_{ij} \right) u
\end{equation}
where $v_{ij} = (v_j-v_i).u$ is the relative velocity of the particles along direction $u$.

\subsubsection{Finite elements}
Finite elements is a powerful paradigm for modeling continuous material. At our level, we can see them as springs acting on more than two points simultaneously. For example, a tetrahedral finite element acts on the four vertices of a tetrahedron and allows a more effective control of stiffness and volume than using springs.


%===========================================================================================

\section{Solids}
A solid is a moving reference frame with a mass matrix. In two dimensions it has three degrees of freedom (DOFs), two translations and one rotation, while in three dimensions it has six DOFs, three translations and three independent rotation values. The remainder of this document focuses on three dimensions.

\subsection{Orientation}
There are different ways of modeling orientation ot one frame with respect to another in three dimensions, each of them with advantages and drawbacks:
\begin{itemize}
\item matrices directly define the axes of the solid with respect to a reference frame. They allow fast projections from one frame to another but they contain nine dependent entries and they can not be set up intuitively;
\item Euler angles are compact (three parameters) and intuitive but they have singularities and they can not be easily combined;
 \item (axis, angle) pairs are more intuitive and more compact (four parameters) than matrices but they do not allow projections and combinations;
\item quaternions are compact (four parameters) and allow easy projections and combinations, but they are not easy to set up intuitively.
\end{itemize}

\subsubsection{Orientation matrices}
Orientations matrices are $3\times 3$ matrices, each column gathering the coordinates of one axis of the rotated frame with respect to the reference frame. Each column is thus a unit vector orthogonal to the others. This creates six relations among the nine parameters, leaving three independent DOFs.

\subsubsection{Euler angles}
Euler angles model a sequence of three rotations along three pairwise-independent directions. For example, the following matrix product represents a rotation $\alpha$ along axis $x$ followed by a rotation $\beta$ along rotated axis $y$ followed by a rotation $\gamma$ along the twice rotated axis $z$.
%\begin{equation}\label{eq:angles euler}
$$
\left(\begin{array}{ccc}
1 & 0 & 0 \\
0 & \cos\alpha & -\sin\alpha \\
0 & \sin\alpha &  \cos\alpha
\end{array}\right)
\left(\begin{array}{ccc}
\sin\beta & 0 &  \cos\beta\\
0 & 1 & 0 \\
\cos\beta & 0 & -\sin\beta
\end{array}\right)
\left(\begin{array}{ccc}
\cos\gamma & -\sin\gamma & 0\\
\sin\gamma & \cos\gamma & 0\\
0 & 0 & 1\\
\end{array}\right)
$$
%\end{equation}
Alternatively, this can be seen as a rotation  $\gamma$ along axis $z$ followed by a rotation $\beta$ along the fixed axis $y$ followed by a rotation $\alpha$ along the fixed axis $x$. An example of singularity is the fact that in this system, rotation $(\pi,\pi,0)$ is equivalent with rotation $(0,0,\pi)$. Another example is the fact that rotation $(\alpha,\pi /2, \gamma)$ is equivalent with rotation $(0,\pi /2, \alpha+\gamma)$ for any $\alpha$ and $\gamma$ (this loss of one DOF is called \emph{gimbal lock}).

\subsubsection{Axis, angle}
The rotation $\theta$ along an axis defined by a unit vector $n$ has the following matrix:
$$
\rot{\theta}{u} = \mat{I} + \sin\theta\oppvec{n} + (1-\cos\theta)\oppvec{n}^2
$$
where matrix $\oppvec{n}$  is the vector product matrix operator: 
%\begin{equation}\label{eq:oppvec}
$$
\oppvec{n} = \left(\begin{array}{ccc}
0 & -n_z & n_y \\
n_z & 0 & -n_x \\
-n_y & n_x & 0
\end{array}\right)
$$
%\end{equation}
It is possible to convert a matrix back to (axis, angle) by noticing that  $\trace{\mat R}=1+2\cos\theta$ et que $\mat R -\transp{\mat R} = 2\sin\theta\oppvec{n}$

\subsubsection{Quaternions}
Quaternions are an extension of complex numbers: $ \bm q = w + x\bm i + y \bm j + z \bm k = (w,\vect v)$.\\
w is the real part, \vect v the imagianry part.
Properties of \bm i, \bm j, \bm k:
$$
\begin{array}{l}
  \bm i^2 = \bm j^2 = \bm k^2 = -1\\
  \bm{ij}=\bm k,\;\bm{ji} = -\bm k\\
  \bm{jk}=\bm i,\;\bm{kj} = -\bm i\\
  \bm{ki}=\bm j,\;\bm{ik} = -\bm j\\
\end{array}
$$
A 3d vector is a pure imaginary quaternion:
$$
\bm p = (0,x,y,z)
$$
Product of quaternions (not commutative):
$$
\bm{q_1q_2} = (w_1w_2 - \bm v_1.\bm v_2, \; w_1\bm v_2 + w_2\bm v_1 + \bm v_1 \wedge \bm v_2)
$$
Conjugate quaternion:
$$
\begin{array}{l}
  \bm{\bar q} = w - x\bm i - y \bm j - z \bm k\\
  \bm{q\bar q} = w^2 + x^2 + y^2 + z^2
\end{array}
$$
Unit quaternions used to model rotations:
$$
\bm{q\bar q} = 1
$$
Rotation $(\theta, \bm u)$: {\em ($\bm u^2=1$)}
$$
\bm q_{(\theta,u)} = ( \cos{\frac{\theta}{2}}, u_x\sin{\frac{\theta}{2}}, u_y\sin{\frac{\theta}{2}}, u_z\sin{\frac{\theta}{2}})
$$
Rotation of a vector \bm p:$\;\;\;\bm{ qp\bar{q} }$\\
Rotation matrix associated with a unit quaternion:
$$
\left( \begin{array}{ccc}
  1 - 2y^2 - 2z^2 & 2xy - 2wz & 2xz + 2wy \\
  2xy + 2wz & 1-2x^2-2z^2 & 2yz - 2wx \\
  2xz - 2wy & 2yz + 2wx & 1 - 2x^2 - 2y^2
\end{array} \right)
$$
Combination of rotations: $\rot{\alpha}{u}\rot{\beta}{v} \longrightarrow q_{(\alpha,u)}q_{(\beta,v)}$\\
Inverse rotation: $\inv{ q_{(\theta,u)}} = q_{(-\theta,u)} = q_{(\theta,-u)} = (-w,\vect v) = (w,\vect -v)$ \\
Conversion $(w, \vect v) \longrightarrow ( \theta, \vect u)$ :
\begin{eqnarray*}
\cos(\theta/2) &=& w\\
\sin(\theta/2) &=& \|\vect v\|\\
\vect u &=& \vect v/\|\vect v\|
\end{eqnarray*}

The time derivative of the unit quaternion $q$ defining the orientation of a solid with angular velocity $\omega$ (vector of $\RRR$, see section \ref{sec:omega}) is: $\dot q = \frac{1}{2}\omega q$.


\subsection{Kinematics}
\todo{choose omega or Omega. Simplify notations where possible.}
\todo{choose n or u.}
\subsubsection{Derivative in \Rep{0} of a vector fixed in \Rep{1}. Angular velocity.} \label{sec:omega}
Consider vector \fixedans{\vect u}{1}, fixed in frame \rep{1}. Frame \rep{1} rotates with respect to frame \rep{0}. Consider the projection \vecin{\fixedans{\vect u}{1}}{0} of this vector to \rep{0}, which we sometimes call \vect u for clarity, and its derivative in \rep{0} which we write $\derivedans{\fixedans{\bm u}{1}}{0}$.


Let \mat{R(dt)} be the rotation of \rep{1} between time $t$ and $t+dt$. We write:
\begin{eqnarray}
 \vect u(t+dt) &=& \mat{R(dt)} \vect u(t)\\
 \vect u(t+dt) - \vect u(t)&=& (\mat{R(dt)}-\ident{}) \vect u(t) \label{eq ri}
\end{eqnarray}
Let $\dot{\theta}$ be the angular velocity along the rotation axis, which we set to \vect z for clarity. The first-order Taylor series is: 
$$
 \mat{R(dt)}-\ident{} = 
 \left(\begin{array}{ccc}
  cos(\dot{\theta}dt)-1 & -sin(\dot{\theta}dt) & 0\\   
   sin(\dot{\theta}dt)& cos(\dot{\theta}dt)-1& 0\\ 
  0 & 0 & 0 
 \end{array}\right)
 \longrightarrow
 \left(\begin{array}{ccc}
  0 & -\dot{\theta}dt & 0\\
  \dot{\theta}dt & 0 & 0\\
  0 & 0 & 0
 \end{array}\right) 
 =
 \dot{\theta}dt \oppvec{z}
$$
which can be easily extended to any rotation axis. Let $\vecrot{1}{0}=\dot{\theta}\vect n$. Dividing expression \ref{eq ri} by $dt$ and decreasing $dt$ to $0$ gives $\dot{\mat R} = \oppvec{ \vecrot{1}{0} }$.

We can write the time derivative in \rep{0}: $\derivedans{\fixedans{\bm u}{1}}{0}  = \vecrot{1}{0} \wedge  \fixedans{\vect u}{1}$, or more simply:

\begin{equation}\label{vrot}
\begin{array}{rcl}
 \dot{\vect u} &=& \dot{\mat R} \vect u \\
               &=& \oppvec{ \vecrot{1}{0} } \vect u \\
               &=& \vecrot{1}{0} \wedge \vect u
\end{array}
\end{equation}
 

\subsection{Velocity in \Rep{0} of a point fixed in \Rep{1}. Velocity field.}
We consider the velocity $\vfdans{A}{1}{0}$ in \rep{0} of a point $A$ fixed in \rep{1} while \rep{1} moves with respect to \rep{0}. Let $O_0$ be the origin of \rep{0} and $O_1$ the origin of \rep{1}. The following relation holds:
$$ \vfdans{A}{1}{0} = \vfdans{O_1}{1}{0} + \vecrot{1}{0} \wedge \vecf{O_1A} \label{eq vit} \label{eq vit solide}
$$

\subsection{Acceleration in \Rep{0} of  point fixed in \Rep{1}. Acceleration field. }
By deriving equation \ref{eq vit}, and based on the fact that $\vecf{O_1A}$ is fixed in \rep{1}, we get the acceleration of A, fixed in \rep{1}, with respect to \rep{0}:
\begin{equation}\label{eq acc}
 \afdans{A}{1}{0} = \afdans{O_1}{1}{0} + \accrot{1}{0}\wedge \vecf{O_1A} + \vecrot{1}{0} \wedge \left( \vecrot{1}{0} \wedge \vecf{O_1A} \right)
\end{equation}

\subsection{Derivative in \rep{0} of a vector defined in \rep{1}}
Let $(\vect e_1, \vect e_e, \vect e_3)$ be a base of \rep{1}. We have:
\begin{eqnarray*}
 \vecin{u}{1} &=& \sum_i x_i \vect e_i\\
 \dot{\vect u} &=& \sum_i \dot x_i \vect e_i + \sum_i x_i \dot{\vect e}_i
\end{eqnarray*}
and thus:
\begin{equation}\label{eq vec mob}
 \derivedans{u}{0} = \derivedans{u}{1} + \vecrot{1}{0} \wedge \vect u
\end{equation}


\subsection{Velocity in \rep{0} of a point moving in \rep{1}.}
Let \vmdans{A}{1} be the velocity of point $A$ with respect to \rep{1}. We have:
\begin{equation}\label{eq vit mob}
\vmdans{A}{0} = \vmdans{A}{1} + \vfdans{O_1}{1}{0} + \vecrot{1}{0} \wedge \vecf{O_1A}
\end{equation}
Note that $O_1$ being the origin of frame \rep{1}, we have $\vmdans{O_1}{0} = \vfdans{O_1}{1}{0}$.

\subsection{Acceleration in \rep{0} of a point moving in \rep{1}. Coriolis acceleration.}
EBy deriving equation \ref{eq vit mob} nous we get:
$$
 \amdans{A}{0} = \underbrace{\amdans{A}{1} + \vecrot{1}{0}\wedge \vmdans{A}{1}}_{\overset{\circ}{\vmdans{A}{1}}} + \amdans{O_1}{0} + \underbrace{\accrot{1}{0}\wedge \vect{O_1}{A} + \vecrot{1}{0}\wedge \vmdans{A}{1} + \vecrot{1}{0} \wedge (\vecrot{1}{0}\wedge \vecf{O_1A})}_{\overset{\circ}{\vecrot{1}{0} \wedge \vecf{O_1A}}}
$$
or:
\begin{equation}\label{eq acc mob}
 \amdans{A}{0} = \amdans{A}{1} +  \amdans{O_1}{0} + \vecrot{1}{0} \wedge (\vecrot{1}{0}\wedge \vecf{O_1A}) + 2\vecrot{1}{0}\wedge \vmdans{A}{1}
\end{equation}
with:
\begin{itemize}
\item $\amdans{A}{1} = \sum_i \ddot x_i \vect e_i$ relative acceleration
\item $\amdans{O_1}{0}$ frame acceleration (?)
\item $\vecrot{1}{0} \wedge (\vecrot{1}{0}\wedge \vecf{O_1A})$ centripetal acceleration
\item $2\vecrot{1}{0}\wedge \vmdans{A}{1}$ Coriolis acceleration
\end{itemize}


\subsection{Dynamics}
Solids accelerate linearly due to forces, and accelerate angularly due to torques. A given torque has the same value everywhere in the solid. However, a given force generates different torques at different points. The torque $\tau$ applied at point $c$ generated by a force $f$ applied at point $b$ is: $\tau=cb\wedge f$.

The acceleration $\ddot c$ of the mass center of a solid and its angular acceleration $\dot \omega$ with respect to the world are given by the relations:

    \begin{eqnarray*}
        m \bf{ \ddot c } = \sum\bf{f_{ext}} \\%\label{eq:PFD1}\\
        \bf{ I_M \dot \omega } + \bm{ \omega \times I_M \omega} = \sum\bf{\tau_{ext}} %\label{eq:PFD2}
    \end{eqnarray*}
where $m$ is the mass of the solid, $\sum {\bf f_{ext}}$ is the sum of the forces applied to the solid, ${\bf I_M}$ is the inertia matrix et $\sum {\bf \tau_{ext}}$ the sum of the torques applied to the solid and expressed at its mass center. The inertia matrix is given by:
$${\bf I_M} = \int_x \int_y \int_z \rho (x,y,z)\begin{bmatrix}  y^2+z^2 & -xy & -xz \\ -xy & x^2+z^2 & -yz \\ -xz & -yz & x^2+y^2 \end{bmatrix} dx dy dz$$
where $\rho(x,y,z)$ is the volumic mass of the naterial (in $kg.m^{-3}$).



% 
% \appendix
\chapter{Appendices} \label{chap:appendices}

\section{Mechanical class diagrams} \label{sec:umlmeca}
Figure \ref{fig:umlMechClasses} shows the mechanical classes of \sofa. Not all members and methods are shown. For a full list, please refer to the source code documentation.

\begin{figure}[htp]
	\hspace{-2cm}
	\includegraphics*[width=20cm]{fig/uml-mechanical-classes.eps}  
\label{fig:umlMechClasses} 
\caption{Class diagram of the mechanics.}
\end{figure}

	%\setlength{\textwidth }{16cm}	% largeur de ligne









\end{document}          
